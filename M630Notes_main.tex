\documentclass[12pt]{report}

\usepackage[margin=1in]{geometry}
\usepackage{amsmath}
\usepackage{amssymb}
\usepackage{tcolorbox}
\usepackage{tikz} % Put letters in circles
\usepackage{bm} % bold math symbols
\usepackage{amsfonts}
\usepackage{framed}
\usepackage[toc,page]{appendix}
\usepackage{amsthm}
\usepackage{graphicx}
\usepackage{bm}
\usepackage{color}   %May be necessary if you want to color links
\usepackage{hyperref}
\hypersetup{
	colorlinks=true, %set true if you want colored links
	linktoc=all,     %set to all if you want both sections and subsections linked
	linkcolor=blue,  %choose some color if you want links to stand out
}

\newtheorem{prop}{Proposition}[section]
\newtheorem{lem}{Lemma}[section]
\newtheorem{define}{Definition}[section]
\newtheorem{axim}{Axiom}
\newtheorem{soln}{Solution}
\newtheorem{thm}{Theorem}
\newtheorem{ex}{Example}[section]
\newtheorem{corollary}{Corollary}[section]

% Make the leftbar red
\renewenvironment{leftbar}[1][\hsize]
{%
	\def\FrameCommand
	{%
		{\color{green}\vrule width 3pt}%
		\hspace{0pt}%must no space.
		\fboxsep=\FrameSep\colorbox{white}%
	}%
	\MakeFramed{\hsize#1\advance\hsize-\width\FrameRestore}%
}
{\endMakeFramed}

% Add encircle command
\newcommand\encircle[1]{%
	\tikz[baseline=(X.base)] 
	\node (X) [draw, shape=circle, inner sep=0] {\strut #1};}


\begin{document}

\tableofcontents

\part{Measure Theory And Lebesgue Integration}
    %\documentclass[12pt]{article}
\chapter{Measure Theory}
%
%\usepackage[margin=1in]{geometry}
%\usepackage{amsmath}
%\usepackage{framed}
%\usepackage{amsthm}
%\usepackage{amsfonts}
%\usepackage{graphicx}
%\usepackage{tcolorbox}
%\usepackage{bm}
%
%\newtheorem{lem}{Lemma}
%\newtheorem{define}{Definition}[section]
%\newtheorem{axim}{Axiom}
%\newtheorem{soln}{Solution}
%\newtheorem{thm}{Theorem}
%\newtheorem{ex}{Example}[section]
%
%\begin{document}


\section{Measurable Sets ($\sigma$-Algebras)}

\begin{leftbar}
\begin{define}
	A \textbf{$\sigma$-Algebra} on a set $X$ is a collection, denote $\mathfrak{A}$, of subsets of $X$ s.t.
	\begin{itemize}
		\item $ \varnothing \in \mathfrak{A}$
		\item If $A \in \mathfrak{A}$, then $A^{c}=X\\A\in\mathfrak{A}$
		\item If $\{A_{i}/i\in\mathbb{N}\}$ is a countable family of sets in $\mathfrak{A}$ then $\cup_{i=1}^{\infty}A_{i}\in\mathfrak{A}$
	\end{itemize}
\end{define}
\end{leftbar}

\begin{leftbar}
\begin{define}
	A \textbf{measurable space} $(X,\mathfrak{A})$ is a set X with a $\sigma$-algebra on X. The elements of that collection are called measurable sets.
\end{define}
\end{leftbar}


\begin{leftbar}
\begin{prop}
	Let a set X and $\mathfrak{A}$ be a $\sigma$-Algebra on X. Then $X\in\mathfrak{A}$ and $\mathfrak{A}$ is closed under countable intersections.
\end{prop}
\end{leftbar}

\begin{proof}
	\begin{itemize}
		\item Since $\varnothing \in \mathfrak{A}$ then $X=\varnothing^{c}\in\mathfrak{A}$
		\item Let $\{A_{i}/i\in\mathbb{N}\}$ be a countable family of elements of $\mathfrak{A}$. Be definition, $\forall i\in\mathbb{N}, A_{i}^{c}\in\mathfrak{A}$. In other words, $\{A_{i}^{c}/i\in\mathbb{N}\}$
	\end{itemize}
	\[\therefore,\hspace{3em} \cap_{i=1}^{\infty}A_{i}=(\cup_{i-1}^{\infty}A_{i}^{c})^{c}\in\mathfrak{A}\]
\end{proof}

\begin{ex}
	The smallest $\sigma$-Algebra that can be defined over an arbitrary set is the empty set $\{\varnothing,X\}$
\end{ex}

\begin{ex}
	The largest $\sigma$-Algebra that can be defined is the power set, $\mathfrak{P}(x)$. The power set is the collection of all possible sets of X.
\end{ex}

\begin{ex}
	Let T be the collection of open sets in X ((X,T) is called a \textbf{Topological Space}). The $\sigma$-Algebra of X generalized by T is called the Borel $\sigma$-Algebra on X. We denote this $\mathfrak{B}(x)$. Its elements are called Borel Sets. 
\end{ex}

\encircle{R} A closed set is the compliment of an open set. It is possible to be closed and open at  the same time. \\
\encircle{R} By definition, the complement of sets in the $\sigma$-Algebra is also in the $\sigma$-Algebra. This means that the closed sets are also in the $\sigma$-Algebra. For instance, the Borel-Algebra on $\mathbb{R}^{n}$ is generated by the collection of cubes, C, of the form $C=(a_{1},b_{1})(a_{2},b_{2})(\dotsm)(a_{n},b_{n})$. \\

\begin{leftbar}
\begin{define}
	A \textbf{measure}, $\bm{\mu}$, on a set X, is a map $\mu:\mathfrak{A}\rightarrow[0,\infty]$ on a $\sigma$-Alembra $\mathfrak{A}$ of X s.t.
	\begin{itemize}
		\item $\mu(\varnothing)=0$
		\item If $\{A_{i}/i\in\mathbb{N}\}$ is a countable family of mutually disjoint sets of $\mathfrak{A}$, ie $A_{i}\cap A_{j}=\varnothing, i\neq j$
	\end{itemize}
    \[
    \mu(\bigcup_{i=1}^{\infty}A_{i})=\sum_{i=1}^{\infty}\mu(A_{i})
    \]
\end{define}
\end{leftbar}

The measure is said to be finite if $\mu(x)<\infty$ and said to be $\sigma$-finite if $\exists\{A_{i}\in/i\in\mathbb{N} \}$ of measurable subsets of X s.t. $\forall i\in\mathbb{N},\mu(A_{i})<\infty$ and $X=\cup_{i=1}^{infty}A_{i}$

%\end{document}
    
    
\part{Homework}
    \include{M630_homeworks}
\end{document}
