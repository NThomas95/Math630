%\documentclass[12pt]{article}
\chapter{Measure Theory}
%
%\usepackage[margin=1in]{geometry}
%\usepackage{amsmath}
%\usepackage{framed}
%\usepackage{amsthm}
%\usepackage{amsfonts}
%\usepackage{graphicx}
%\usepackage{tcolorbox}
%\usepackage{bm}
%
%\newtheorem{lem}{Lemma}
%\newtheorem{define}{Definition}[section]
%\newtheorem{axim}{Axiom}
%\newtheorem{soln}{Solution}
%\newtheorem{thm}{Theorem}
%\newtheorem{ex}{Example}[section]
%
%\begin{document}


\section{Measurable Sets ($\sigma$-Algebras)}

\begin{leftbar}
\begin{define}
	A \textbf{$\sigma$-Algebra} on a set $X$ is a collection, denote $\mathfrak{A}$, of subsets of $X$ s.t.
	\begin{itemize}
		\item $ \varnothing \in \mathfrak{A}$
		\item If $A \in \mathfrak{A}$, then $A^{c}=X\\A\in\mathfrak{A}$
		\item If $\{A_{i}/i\in\mathbb{N}\}$ is a countable family of sets in $\mathfrak{A}$ then $\cup_{i=1}^{\infty}A_{i}\in\mathfrak{A}$
	\end{itemize}
\end{define}
\end{leftbar}

\begin{leftbar}
\begin{define}
	A \textbf{measurable space} $(X,\mathfrak{A})$ is a set X with a $\sigma$-algebra on X. The elements of that collection are called measurable sets.
\end{define}
\end{leftbar}


\begin{leftbar}
\begin{prop}
	Let a set X and $\mathfrak{A}$ be a $\sigma$-Algebra on X. Then $X\in\mathfrak{A}$ and $\mathfrak{A}$ is closed under countable intersections.
\end{prop}
\end{leftbar}

\begin{proof}
	\begin{itemize}
		\item Since $\varnothing \in \mathfrak{A}$ then $X=\varnothing^{c}\in\mathfrak{A}$
		\item Let $\{A_{i}/i\in\mathbb{N}\}$ be a countable family of elements of $\mathfrak{A}$. Be definition, $\forall i\in\mathbb{N}, A_{i}^{c}\in\mathfrak{A}$. In other words, $\{A_{i}^{c}/i\in\mathbb{N}\}$
	\end{itemize}
	\[\therefore,\hspace{3em} \cap_{i=1}^{\infty}A_{i}=(\cup_{i=1}^{\infty}A_{i}^{c})^{c}\in\mathfrak{A}\]
\end{proof}

\begin{ex}
	The smallest $\sigma$-Algebra that can be defined over an arbitrary set is the empty set $\{\varnothing,X\}$
\end{ex}

\begin{ex}
	The largest $\sigma$-Algebra that can be defined is the power set, $\mathfrak{P}(x)$. The power set is the collection of all possible sets of X.
\end{ex}

\begin{ex}
	Let T be the collection of open sets in X ((X,T) is called a \textbf{Topological Space}). The $\sigma$-Algebra of X generalized by T is called the Borel $\sigma$-Algebra on X. We denote this $\mathfrak{B}(x)$. Its elements are called Borel Sets. 
\end{ex}

\encircle{R} A closed set is the compliment of an open set. It is possible to be closed and open at  the same time. \\
\encircle{R} By definition, the complement of sets in the $\sigma$-Algebra is also in the $\sigma$-Algebra. This means that the closed sets are also in the $\sigma$-Algebra. For instance, the Borel-Algebra on $\mathbb{R}^{n}$ is generated by the collection of cubes, C, of the form $C=(a_{1},b_{1})(a_{2},b_{2})(\dotsm)(a_{n},b_{n})$. \\

\begin{leftbar}
\begin{define}
	A \textbf{measure}, $\bm{\mu}$, on a set X, is a map $\mu:\mathfrak{A}\rightarrow[0,\infty]$ on a $\sigma$-Alembra $\mathfrak{A}$ of X s.t.
	\begin{itemize}
		\item $\mu(\varnothing)=0$
		\item If $\{A_{i}/i\in\mathbb{N}\}$ is a countable family of mutually disjoint sets of $\mathfrak{A}$, ie $A_{i}\cap A_{j}=\varnothing, i\neq j$
	\end{itemize}
    \[
    \mu(\bigcup_{i=1}^{\infty}A_{i})=\sum_{i=1}^{\infty}\mu(A_{i})
    \]
\end{define}
\end{leftbar}

The measure is said to be finite if $\mu(x)<\infty$ and said to be $\sigma$-finite if $\exists\{A_{i}\in/i\in\mathbb{N} \}$ of measurable subsets of X s.t. $\forall i\in\mathbb{N},\mu(A_{i})<\infty$ and $X=\cup_{i=1}^{infty}A_{i}$

\begin{leftbar}
\begin{define}
	A measure space is a triple (X, $\mathfrak{A}$, $\mu$) 
\end{define}
\end{leftbar}

\begin{ex}
	The counting measure, $\nu$, on X is defined by $\nu(x)=$\{The number of elements in X\}. With convention $\nu(x)=\infty$ if x is an infinite set.
\end{ex}

\begin{ex}
	The \textbf{delta measure}, $\delta_{x_{0}}$, where $x_{0}\in\mathbb{R}$ on the Borel-alg of $\mathbb{R}$ is defined by:
	
	\[
	\delta_{x_{0}} = \begin{cases}
					  1 \hspace{3em}if x_{0}\in A\\
					  0 \hspace{3em}if x_{0}\notin A
	\end{cases}
	\]
\end{ex}

\begin{ex}
	The \textbf{Lebesgue Measure}. We saw the Borel Algebra on $\mathbb{R}$ is generated by the cubes $(a_{1},b_{1})(a_{2},b_{2})\dots(a_{n},b_{n})$. The Lebesague Measure, $\lambda$ on the Borel Algebra s.t. $\lambda(c)=$ volume(c)$=(b_{1}-a_{1})(b_{2}-a_{2})\dots(b_{n}-a_{n})$.
\end{ex}

\encircle{R} The Lebesgue Measure is $\sigma-finite$, indeed $\mathbb{R}=\cup_{i=1}^{\infty}(-i,i)^{n}$

\begin{leftbar}
\begin{thm}
	A subset of $A\in\mathbb{R}^{n}$ is Lebesgue Measurable if and only if $\forall \epsilon > 0,$ $\exists$ F closed, $\exists$ G open, $F\subset A \subset G$, $\lambda(G\setminus F)<\epsilon$.\\
	
	Moreover: 
	\[
	\begin{aligned}
	\lambda(A) &= \inf \{\lambda(u) | \text{u is open and } A\subset u \}\\
	\lambda(A) &= \sup \{\lambda(u) | \text{k is closed and } k\subset A \}
	\end{aligned}
	\]
\end{thm}
\end{leftbar}

\begin{leftbar}
\begin{prop}
	The Lebesgue Measure is translation invariant. ie
	
	\[
	\lambda(\mathfrak{T}_{h}A) = \lambda(A)\hspace{3em}
	\]
	
	Where
	
	\[
	\textrm{T}_{h}A = \{y\in\mathbb{R}^{n} | \exists x\in A, y=x+h \}
	\]
\end{prop}
\end{leftbar}

\begin{leftbar}
\begin{prop}
	Let $T:\mathbb{R}^{n}\rightarrow\mathbb{R}^{n}$ be a linear mapping. We denote $TA = \{y\in\mathbb{R}^{n} | \exists x\in A, y=Tx \}$. Then $\lambda(TA)=|det(T)|\lambda(A)$
\end{prop}
\end{leftbar}

\begin{leftbar}
\begin{corollary}
	The Lebesgue Measure is invariant by rotation and has the scaling property. ie $\forall t>0, \lambda(tA) = t^{n}\lambda(A)$
\end{corollary}
\end{leftbar}

\begin{proof}
	A rotation R and scaling T are linear transformations. Further, $|det(R)|=1$ and $|det(T)|=t^{n}$.
\end{proof}

\begin{leftbar}
\begin{define}
	Let (X,$\mathfrak{A},\mu$) be a measure space. A subset $A\subset \mathfrak{A}$ is said to have a measure zero if it is measurable and the measure $\mu(A)=0$.
\end{define}
\end{leftbar}

\begin{leftbar}
\begin{prop}
	A singleton in $\mathbb{R}$ has a measure zero. ie $\forall x\in\mathbb{R}$, $\lambda(\{x\})=0$.
\end{prop}
\end{leftbar}

\begin{proof}
	From the previous theorem we know that:
	
	\[
	\begin{aligned}
	\lambda(\{x\}) &= \inf\{\lambda(u) | \text{u is open and } \{x\}\subset u\}\\
	               &= \lim_{\epsilon\rightarrow0^{+}}\lambda(\{y/ |x-y|<\epsilon \})\\
	               &= \lim_{\epsilon\rightarrow0^{+}}2\epsilon\\
	               &= 0
	\end{aligned}
	\]
\end{proof}

\begin{leftbar}
\begin{corollary}
	Every countable subset $A=\{x_{i}\in\mathbb{R}/i\in\mathbb{N}\}$ of $\mathbb{R}$ has measure 0.
\end{corollary}
\end{leftbar}

\begin{proof}
	Let $A=\cup_{i=1}^{\infty}A_{i}$. Then, using the properties of measures, we have:
	
	\[
	\begin{aligned}
	\lambda(A) &= \lambda(\bigcup_{i=1}^{\infty}A_{i})\\
	           &= \sum_{i=1}^{\infty}\lambda(A_{i})\\
	           &= 0
	\end{aligned}
	\]
\end{proof}


























%\end{document}